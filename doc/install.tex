\documentclass[pdftex,12pt,a4paper]{article}
\renewcommand{\abstractname}{About this package}

\usepackage[pdftex]{graphicx}
\usepackage[centering,includeheadfoot,margin=2cm]{geometry}
\usepackage[T1]{fontenc}
\usepackage[utf8]{inputenc}
\usepackage{lmodern}
\usepackage{relsize}
\usepackage{microtype}
\usepackage{url}
\usepackage{dirtree}
\usepackage{pifont}
\usepackage{listings}
\usepackage{courier}
\usepackage{framed}
\usepackage{makeidx}
\usepackage[usenames,dvipsnames]{color}
\usepackage{hyperref}
\usepackage[x11names, rgb]{xcolor}
\usepackage{tikz}
\usetikzlibrary{snakes,arrows,shapes}
\usepackage{amsmath}

\definecolor{orange}{rgb}{1,0.5,0}
\definecolor{lightgray}{RGB}{249,249,249}
\definecolor{darkgray}{RGB}{69,69,69}
\definecolor{darkblue}{RGB}{29,88,145}

\makeindex

% For code snippets
\lstset{%
basicstyle=\scriptsize\ttfamily,
stringstyle=\color{red},
commentstyle=\itshape\color{darkblue},
numbersep=5pt,
extendedchars=true,
backgroundcolor=\color{lightgray},
xleftmargin=17pt,
framexleftmargin=17pt,
framexrightmargin=5pt,
framexbottommargin=4pt,
showspaces=false,
showstringspaces=false,
showtabs=false,
frame=b,
tabsize=2,
captionpos=t,
breaklines=true,
breakatwhitespace=false%
}

\usepackage{caption}
\DeclareCaptionFont{white}{\color{white}}
\DeclareCaptionFormat{listing}{\colorbox[cmyk]{0.43, 0.35, 0.35,0.01}{\parbox{\textwidth}{\hspace{15pt}#1#2#3}}}
\captionsetup[lstlisting]{format=listing,labelfont=white,textfont=white, singlelinecheck=false, margin=0pt, font={bf,footnotesize}}

\newcommand{\treedesc}[1]{\textsf{\scriptsize{#1}}}
% For text describing a variable value placeholder without a defined value
\newcommand{\uservariable}[1]{\fbox{``\textit{#1}''}\linebreak[0]}
% For variable definitions defined by the framework
\newcommand{\variable}[1]{\mbox{\texttt{#1}}\linebreak[0]}
% For variable values defined by the framework
\newcommand{\varvalue}[1]{\mbox{\color{orange}#1}\linebreak[0]}
% For words that need indexing
\newcommand{\keyword}[1]{\mbox{\textmd{{#1}}}\index{#1}\linebreak[0]}

% For PHP classnames
\newcommand{\classname}[1]{\mbox{\color{darkblue}\texttt{#1}}\linebreak[0]}
% For PHP Constants
\newcommand{\constant}[1]{\mbox{\color{red}\texttt{#1}}\linebreak[0]}
% For PHP inline code snippets
\newcommand{\snippet}[1]{\lstset{basicstyle=\ttfamily\small,language=PHP}\lstinline!#1!\lstset{basicstyle=\ttfamily\scriptsize}\linebreak[0]}

% For filenames
\newcommand{\filename}[1]{\mbox{\color{darkgray}\textit{#1}}\linebreak[0]}
% For document names
\newcommand{\docref}[1]{\mbox{\ding{226}\textsl{#1}}\linebreak[0]}

\renewcommand{\labelitemi}{\ding{110}}
%\newcommand{\implies}{\ding{229}}


% For End of document
\newcommand{\eod}{\vspace{\stretch{1}}\centering{\textbf{[END OF DOCUMENT]}}}

% For latin abbreviations
\newcommand{\eg}{e.g.\ }
\newcommand{\ie}{i.e.\ }

% For notes
\newenvironment{note}
{\begin{center}\textcircled{i}\hspace{1ex}%
 \begin{minipage}{0.8\columnwidth}%
 \begin{framed}%
 \begin{small}%
}
{\end{small}%
 \end{framed}%
 \end{minipage}%
 \end{center}%
}

% Prevent page breaking
\newenvironment{unbreakable}{\samepage}{\pagebreak[0]}

\hypersetup{
    bookmarks=true,         % show bookmarks bar?
    unicode=false,          % non-Latin characters in Acrobat’s bookmarks
    pdftoolbar=true,        % show Acrobat’s toolbar?
    pdfmenubar=true,        % show Acrobat’s menu?
    pdffitwindow=false,     % window fit to page when opened
    pdfstartview={FitH},    % fits the width of the page to the window
    colorlinks=true,        % false: boxed links; true: colored links
    linktoc=page,
    linkcolor=red,          % color of internal links
    citecolor=green,        % color of links to bibliography
    filecolor=magenta,      % color of file links
    urlcolor=blue           % color of external links
}


\begin{document}

\title{Simple MVC Framework - Install guide}
\author{Yoann Mikami}
\date{\today}
\maketitle

\begin{abstract}
This package contains the files needed to build a web application using
an \keyword{MVC}\footnote{(Model-View-Controller)} oriented architecture.

This PDF describes the steps needed to setup the initial environment needed
to develop a new web application. This can also be used as a reference for existing sites as well.
\end{abstract}

\tableofcontents

\cleardoublepage

\section{Installation}

\subsection{Requirements}

\begin{description}
    \item[OS] \hfill \\
    Linux (untested on Windows OSes)
    \item[Software] \hfill \\
    \begin{itemize}
        \item PHP \textgreater{} 5.2
        \item Apache 2.x
        \item MySQL 5.x or MongoDB 1.x
    \end{itemize}
\end{description}

\subsection{Dependencies}
Below are the 3rd party packages/frameworks used by this framework.
They should all be supplied under the \filename{opt/} directory so nothing is needed.

\begin{description}
    \item[\url[WideImage]{http://wideimage.sourceforge.net/}] \hfill \\
OO Image manipulation framework. Used by \classname{Utils\_ResourceLoader\_Image}. Licence : \href{http://wideimage.sourceforge.net/license/}{GNU LGPL 2.1}
    \item[\url[savant3]{http://phpsavant.com}] \hfill \\
Savant3 templating engine - not maintained anymore. Licence: GNU LGPL
    \item[\url[spyc]{https://github.com/mustangostang/spyc/}] \hfill \\
Simple PHP YAML Class : YAML parser. Licence : MIT
    \item[\url[Zend]{http://framework.zend.com}] \hfill \\
Zend Framework (partial, 1.12). Used packages :
        \begin{itemize}
            \item{Cache}
            \item{Db}
            \item{Json}
            \item{Loader}
            \item{Locale}
            \item{Log}
            \item{Registry}
            \item{Session}
            \item{Translate}
        \end{itemize}
\end{description}

\subsection{Recommended structure for a web application}
The developer is free to choose where everything is stored under the server \keyword{docroot}.
The following is the recommendation (folder name can be different for each module) :
\begin{description}
    \item[framework] \hfill \\
This framework. May be shared
    \item[my\_site] \hfill \\
The web application specific files (models, controllers, views, layouts, scripts, css\ldots)
    \item[my\_site\_resources] \hfill \\
Files handled by the server, user supplied resources (application logs, cached files, uploaded files\ldots)
    \item[docroot] \hfill \\
The entry point to web applications
\end{description}

These four subparts can either be located at a single location (not recommended), or split (recommended).
Split rationales :
\begin{description}
    \item[framework] \hfill \\
This framework serves as the base for a web application, and thus can be used by many different ones. \\
Putting this framework under its specific location allows to share the same code among several web applications installed on the same server. \\
See the \docref{Developer guide} for the framework structure.
    \item[my\_site] \hfill \\
Separating the site core files (managed by the webmaster/developer) from the server/user resources
    \item[resources] \hfill \\
Eases maintenance; the site files can be packaged into a \emph{tarball} and extracted to the proper location, without risking to overwrite files that are not part of it. Logs and cached files can be preserved easily as well.
server resources folders usually have special folder permissions and thus is more easily managed when it is separated.
    \item[\keyword{docroot}] \hfill \\
For security purposes, \keyword{docroot} should have its own directory with a small \keyword{bootstrap} needed to start the application. \\
The intialization class \classname{Init} (in this framework) can be called by the \keyword{bootstrap} by pointing to its location (outside \keyword{docroot}). \\
This setup assumes a single \keyword{docroot} is shared among all web applications installed on the server. This can be changed by creating an additional folder \uservariable{my\_site} which would contain folders \uservariable{my\_site}, \filename{resources} and \filename{docroot}.
\end{description}

This framework uses ``application \keyword{namespace} concept'' to separate the different web application contents. (This concept is explained in more details later)
An application \keyword{namespace} allows inter-related applications to share the same workspace. A typical use of \keyword{namespace} would be a web site
with a \keyword{front-end} (typically read only) and a \keyword{back-end} (\eg the admin interface). Both have shared resources, shared assets, yet some differences may
be needed at some specific locations.
\begin{note}
Namespacing should not be used for two UNRELATED web applications, as most resources are \emph{shared}, ie. both applications can access
each other's assets, which is not a good idea in general in this use case. A different folder (\eg \uservariable{my\_site2}) should be created instead.
\end{note}

\subsection{Creating the application structure}
A web application should have the following structure :

\begin{unbreakable}
\dirtree{%
.1 \uservariable{site root dir}.
.2 config
    \treedesc{The web application config file (Recommended : \snippet{chmod 0775})}.
.2 controllers
    \treedesc{location for controllers used by the web application}.
.2 langs
    \treedesc{language files (\keyword{gettext})}.
.2 layouts
    \treedesc{location for template files}.
.2 libs
    \treedesc{PHP classes}.
.3 opt
    \treedesc{Application specific 3rd party tools}.
.3 plugins
    \treedesc{init, view, resource plugins}.
.4 Init
    \treedesc{Init plugins folder}.
.4 View
    \treedesc{View plugins folder}.
.4 Resource
    \treedesc{Resource plugins folder}.
.3 savant3.
.4 Savant3.
.5 resources
    \treedesc{Application specific Savant3 template filters and plugins folder}.
.2 models
    \treedesc{location for models used by the web application}.
.2 skins
    \treedesc{Site resources : graphics, \keyword{CSS}, \keyword{javascript}}.
.2 views
    \treedesc{location for view files used by the web application}.
}
\end{unbreakable}

\subsection{Creating the server resources structure}
When separated from the site structure, the server resources should be structured like the following :

\begin{unbreakable}
\dirtree{%
.1 \uservariable{Resources root dir}.
.2 cache
    \treedesc{cached files (if caching is enabled) \\
    (Recommended : \snippet{chmod 2775}, \snippet{chown www-data:www-data})}.
.2 logs
    \treedesc{The web application logs \\
    (Recommended : \snippet{chmod 2775}, \snippet{chown www-data:www-data})}.
.2 upload
    \treedesc{user uploaded files \\
    (Recommended : \snippet{chmod 0775})}.
}
\end{unbreakable}

Server resources and application structures can be merged if there is no need for separating them.
If \filename{resources} folder is external to the \uservariable{site root dir}, the latter should have a symlink pointing at the former :
\begin{unbreakable}
\dirtree{%
.1 \uservariable{site root dir}.
.2 (\ldots).
.2 resources
    \treedesc{\ding{213} [\uservariable{resources root dir}]}.
.2 (\ldots).
}
\end{unbreakable}

It is also possible to mix \emph{symlink} with physical folders inside a \filename{resources} folder inside \uservariable{site root dir} :
\begin{unbreakable}
\dirtree{%
.1 \uservariable{site root dir}.
.2 (\ldots).
.2 resources
    \treedesc{\uservariable{resources root dir}}.
.3 cache
    \treedesc{Cache folder (physical)}.
.3 logs
    \treedesc{Log folder (physical)}.
.3 upload
    \treedesc{\ding{213} [\uservariable{path to external upload folder}]}.
}
\end{unbreakable}

\section{Setup application namespace(s)}

\subsection{About namespaces}
Through this framework, every web application have at least one \keyword{namespace} : the application itself.
By default, a \keyword{namespace} is defined as the server name the application runs onto. \\
For instance, the default application \keyword{namespace} for \url{http://www.my_site.com/} would be \varvalue{www.my\_site.com}.
Namespaces are heavily used in the framework;

Namespaces allow a logical separation of web applications while sharing physical resources.
How an application is namespaced and how many it has has to identified by the developer.

\subsubsection{Use cases}
    \begin{itemize}
        \item A typical 2 \keyword{namespaces} web application is one with front/back ends.
Thus, \url{http://www.my_site.com} would be the \keyword{front-end}, and \url{http://admin.my_site.com} the \keyword{back-end}.
However, the code is shared for both of them, unless one \keyword{override}s the other.

\begin{note}
  Apache \keyword{mod\_rewrite} is required to allow namespacing to work.
\end{note}
        \item An intranet must be accessed through different credentials depending on the user role : \\
            - Recruiter, Interviewer, Administrator \\
  Each of these could yield to 3 application \keyword{namespace}s, such as \varvalue{recruiter.my\_site}, \varvalue{itw.my\_site}, \varvalue{admin.my\_site}.
  Each \keyword{namespace} would then provide different functionalities while keeping a global layout, sharing common features\ldots \\
    While this could be used to apply simple permission control system, application namespacing does not
    replace a user permissions engine such as \keyword{ACL}\footnote{Access Control List} or \keyword{Role Based Access Control}.
        \item An application is deployment in a development server, staging, production server that all require different settings.
  The config file uses application \keyword{namespace}s so each can use common settings while overridding whatever's needed.
  In such a case, the application would exactly be the same for all \keyword{namespace}s, which would be used solely for configuration.
        \item A web site has a PC and mobile version : both share the same resources and text, while the layout may differ to
  address smaller screen sizes.\\
  Two \keyword{namespace}s, mobile and pc can achieve this : page layout can be customized, while the content is still shared.
\end{itemize}

\subsubsection{Namespace overridding}
Application \keyword{namespace}s rely on \keyword{inheritance}.

To allow different \keyword{namespace}s to share common resources, one must inherit another.
For case 1 above, admin \keyword{namespace} would thus inherit the front \keyword{namespace}.
The other way round would not work as the former probably has more features than the latter,
and inheritance allows to \keyword{override} the parent's assets or add your own, but not disable values set by the parent.\\
For case 4, mobile would inherit pc.

\begin{note}
The top application \keyword{namespace} should \emph{always} inherit ``\varvalue{default}'' in the configuration file.
\end{note}

When two or more \keyword{namespace}s are defined, the following process is done when accessing any namespaced resource.
Knowing this allows to determine which \keyword{namespace} should inherit which :

Three \keyword{namespace}s are defined (excl. ``\varvalue{default}'') :
\begin{center}
\begin{unbreakable}
{\scriptsize
\begin{verbatim}
	baz --- overrides ---> bar --- overrides ---> foo --- overrides ---> default
\end{verbatim}
}
\end{unbreakable}
\end{center}

When requesting resource \variable{foobar} :
\begin{itemize}
    \item Is \varvalue{foobar} resource available under \keyword{namespace} \varvalue{baz} ? If YES, then use it $\implies$ [END]
    \item Is \varvalue{foobar} resource available under \keyword{namespace} \varvalue{bar} ? If YES, then use it $\implies$ [END]
    \item Is \varvalue{foobar} resource available under \keyword{namespace} \varvalue{foo} ? If YES, then use it $\implies$ [END]
    \item Resource not found $\implies$ [END]
\end{itemize}
\begin{note}
The ``\varvalue{default}'' \keyword{namespace} is not used during the process.\\
\emph{Exception}: \keyword{skin} files also look for the ``\varvalue{default}'' \keyword{namespace} as well. Non-namespace specific resources can be placed under it.
It is only used to define top-level directories for the various resources the application can access.
Under each of these, \keyword{namespace}s should be defined.
\end{note}

\subsection{Define namespaces in the web application}
For each \keyword{namespace} the application is expected to have (more can be added anytime), a folder named after the \keyword{namespace}
should be created in the following folders :
\begin{unbreakable}
\dirtree{%
.1 Site root dir.
.2 config.
.3 controllers.
.4 \uservariable{namespace}.
.3 langs.
.4 \uservariable{namespace}.
.3 layouts.
.4 \uservariable{namespace}.
.3 libs.
.3 models.
.4 \uservariable{namespace}.
.3 skins.
.4 \varvalue{default}.
.4 \uservariable{namespace}.
.3 views.
.4 \uservariable{namespace}.
}
\end{unbreakable}

\begin{itemize}
    \item \filename{config/config.yaml} contains the configuration for all \keyword{namespace}s in a single file and thus is not namespaced.
    \item \filename{libs} is also shared among all \keyword{namespace}s.
\end{itemize}

Folders inside \varvalue{resources} are not namespaced using folders :
\begin{itemize}
    \item Under \filename{logs/}, \filename{cache/} generated files are prefixed with the application \keyword{namespace}.
    \item \filename{upload} is not namespaced at all, and all user uploaded files are common to all \keyword{namespace}s. \\
    For \keyword{namespace} specific images\ldots the application skins folder should be used instead.
\end{itemize}


\section{Configuring the web application}

The configuration file (a sample is available in this framework's \filename{config-default} folder)
\filename{config.yaml} should be created under the \uservariable{site root dir} \filename{config/} directory.
the \keyword{namespace} ``\varvalue{default}'' is required, thus should always be defined; each \keyword{namespace} the web application
uses should then have a top definition the same way ``\varvalue{default}'' is defined in the sample \filename{config.yaml}.
When a \keyword{namespace} inherits (or \keyword{override}s) another, the \variable{overrides: \uservariable{parent}} config definition should be used.
Each \keyword{namespace} configuration definition only need to (re)define configuration properties that are different (or missing) from its parent.
Refer to the \docref{Developer guide} for the configuration parameters.


\section{Creating the bootstrap file}

Each \keyword{namespace} should have its own \keyword{bootstrap} file located under the \keyword{docroot} folder for the application.
\filename{.htaccess} (or apache vhost config) may be used to dispatch based on specific conditions which \keyword{bootstrap} file to load.
The \keyword{bootstrap} could by in any language; however, a PHP one is recommended to allow easier interaction with this framework. \\
The \keyword{bootstrap} only needs to do one action : call this framework's entry point, \snippet{Init::main()}.
It also needs to define some constants the framework assumes to be defined before init is started :
\begin{description}
 \item[\constant{DATA\_DIR}] [Required] \hfill \\
    Absolute path to the \uservariable{site root dir}
 \item[\constant{LIB\_DIR}] [Required] \hfill \\
    Absolute path to the \uservariable{framework root dir}
 \item[\constant{BOOTSTRAP\_FILE}] [Required] \hfill \\
    Filename of the bootstrap file called; namely, the file being processed. Typically \snippet{basename(__FILE__)} when called from within the bootstrap file itself.
 \item[\constant{NAMESPACE}] [Optional] \hfill \\
    Defaults to \snippet{\$\_SERVER['SERVER\_NAME']} if not defined : \keyword{namespace} to use. \\
    \keyword{Namespace} is mapped to the same record in the config file, as well as \keyword{namespace} folders inside each \uservariable{site root dir} folder. \\
    \begin{note}
    `.' and `-' characters are transformed into `\_' : thus, \keyword{namespace} defined in
    \filename{config.yaml} should be named accordingly, \eg \varvalue{www.mysite.com} shall become \varvalue{www\_mysite\_com}.\\
     `-' is transformed into '\_' for \keyword{namespace} folders, and the latter should thus be named accordingly, \eg \varvalue{www.mysite-admin.com} shall become \varvalue{www.mysite\_admin.com}
    \end{note}
\end{description}
$\implies$ This framework's \keyword{docroot} folder contains a sample PHP bootstrap named \filename{namespace.php}.

\section{Creating the .htaccess file}

Depending on where the \keyword{webroot} points at (application's \variable{docroot} directly or somewhere above, \eg httpdocs), the content
will be different; however, here are a few notes that should be applicable on most cases :
\begin{itemize}
    \item Use \keyword{mod\_rewrite} to reroute ALL requests targeting a specific server / domain to the php newly created \keyword{bootstrap} file.
    Different \keyword{namespace}s should be routed to separate \keyword{bootstrap}s (though not enforced).
    \item If \uservariable{site root dir} is under \variable{docroot}, disallow access to it : only the \keyword{bootstrap} can access it through PHP \snippet{include}s.
    \item The \filename{.htaccess} file can be either in the same folder as the \keyword{bootstrap} files, or somewhere else.
    the \snippet{RewriteRule} should match the chosen organization accordingly.
\end{itemize}


\section{Application ready. good luck!}

Once the system is setup and the basic installation is done, it's time to add content to the application.
\docref{Developer guide} and this framework \emph{PHPDoc} generated documentation should contain everything needed to start build the application.

\cleardoublepage

\printindex

\eod
\end{document}

